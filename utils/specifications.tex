\documentclass[a4paper,8pt]{report}

\usepackage[francais]{babel}
\usepackage[T1]{fontenc}
\usepackage[latin1]{inputenc}

\usepackage[left=2cm]{geometry}
\usepackage{amsmath,amssymb,mathrsfs}
\usepackage{graphicx}
\usepackage[table,xcdraw]{xcolor}

\usepackage{enumerate}

%Utilisation des codes sources en C
\usepackage{listings}
\lstset{
  language=C,
  language=Bash,
  basicstyle=\footnotesize,
  numbers=left,
  numberstyle=\normalsize,
  numbersep=7pt,
}

\title{{\LARGE Kilobot \\ Cahier des charges}}
\author{GUERMONT Arnaud, BIELLE Benjamin}

\begin{document}

\maketitle
\renewcommand{\contentsname}{Sommaire}
\tableofcontents

%%%%%%%%%%%%%%%%%%%%%%%%%%%%%%%%%%%%%%%%%%%%%%%%%%%%%%%%%%%%%%%%%%
%                                                                %
%                          INTRODUCTION                          %
%                                                                %
%%%%%%%%%%%%%%%%%%%%%%%%%%%%%%%%%%%%%%%%%%%%%%%%%%%%%%%%%%%%%%%%%%
\chapter{Introduction}

\noindent \textbf{Maitrise d'oeuvre :} DUBOIS Swan.\\
\noindent \textbf{Maitrise d'ouvrage :} GUERMONT Arnaud, BIELLE Benjamin.\\

\section*{Contexte}\label{sec:name}
\addcontentsline{toc}{section}{Contexte}

\textbf{Kilobot} est une plate-forme constitu\'ee de petits robots autonomes. Cette plate-forme est propos\'ee par l'universit\'e d'Harvard\footnote{http://www.eecs.harvard.edu/ssr/projects/progSA/kilobot.html}. \\
Ces robots se d\'eplacent uniquement par vibration et communiquent par infrarouges.\\

\smallskip
Le but de ce projet est d'impl\'ementer une \textbf{API} permettant de manipuler des mod\`eles robotiques sur la plate-forme Kilobot. Ces mod\`eles ne sont pas pr\'evus pour la plate-forme Kilobot, notre API devra rendre cette impl\'ementation possible en restant simple et si possible facilement extensible.\\

\section*{L'existant}\label{sec:name}
\addcontentsline{toc}{section}{L'existant}

Il existe deux API de base pour la plate-forme Kilobot.\\

\begin{itemize}
\item \textit{\textbf{K-team}}\footnote{http://www.k-team.com/mobile-robotics-products/kilobot}.
\item \textit{\textbf{Kilobotics}}\footnote{https://www.kilobotics.com/}.
\end{itemize}

\medskip
Celle propos\'ee par \textit{\textbf{K-team}} (fournie par d\'efaut avec les robots) nous permet une moins grande libert\'e que celle de \textit{\textbf{Kilobotics}}.\\
Donc notre \textit{\textbf{API}} se basera sur celle-ci car elle nous garantie une plus grande libert\'e, portabilit\'e et extensibilit\'e.

\section*{Veille Technologique}\label{sec:name}
\addcontentsline{toc}{section}{Veille Technologique}

Il existe plusieurs projets similaires aux kilobots, par exemple le projet \textit{\textbf{I-Swarm}}\footnote{http://www.hizook.com/blog/2009/08/29/i-swarm-micro-robots-realized-impressive-full-system-integration}, mais seul la plate-forme \textit{\textbf{Kilobot}} permet de manipuler des centaines de robots de mani\`ere r\'ealiste hors d'un simulateur.\\

%%%%%%%%%%%%%%%%%%%%%%%%%%%%%%%%%%%%%%%%%%%%%%%%%%%%%%%%%%%%%%%%%%
%                                                                %
%                           LE PRODUIT                           %
%                                                                %
%%%%%%%%%%%%%%%%%%%%%%%%%%%%%%%%%%%%%%%%%%%%%%%%%%%%%%%%%%%%%%%%%%
\chapter{Le Produit}

\section*{Description des attentes}\label{sec:name}
\addcontentsline{toc}{section}{Description des attentes}

Le but de ce projet est de prendre en main la plate-forme Kilobot (l'API et les robots) puis de cr\'eer une API manipulant un autre type de mod\`ele robotique.\\

\smallskip
L'objectif de cette API est de retransmettre un mod\`ele robotique (compl\`etement diff\'erent des kilobots) sur cette plate-forme, afin de faciliter la mise en place d'algorithmes sur ce mod\`ele.\\

\smallskip
L'API doit fonctionner sur tous les types de syst\`emes (\textit{Windows}, \textit{OSX}, \textit{Linux}) et doit \^etre extensible afin de facilit\'e l'ajout de futures fonctionnalit\'ees.\\

\section*{Fonctions de l'API}\label{sec:name}
\addcontentsline{toc}{section}{Fonctions de l'API}

\textit{A COMPLETER}

%%%%%%%%%%%%%%%%%%%%%%%%%%%%%%%%%%%%%%%%%%%%%%%%%%%%%%%%%%%%%%%%%%
%                                                                %
%                          CONTRAINTES                           %
%                                                                %
%%%%%%%%%%%%%%%%%%%%%%%%%%%%%%%%%%%%%%%%%%%%%%%%%%%%%%%%%%%%%%%%%%
\chapter{Contraintes}

\section*{Mat\'erielles}\label{sec:name}
\addcontentsline{toc}{section}{Mat\'erielles}

Contrairement au mod\`ele que notre API doit impl\'ementer, les kilobots ne peuvent pas se voir au sens propre, ils ne peuvent communiquer qu'\`a sept centim\`etre atour d'eux.

\section*{Logicielles}\label{sec:name}
\addcontentsline{toc}{section}{Logicielles}

La plate-forme fournie par d\'efaut avec les kilobots n'est pas protable sur \textit{OSX} et \textit{Linux}.\\
Donc cela nous force \`a remplacer cette API par celle de Harvard (\textit{\textbf{Kilobotics}}) permettant de r\'egler ce probl\`eme de portabilit\'e.

\section*{de temps}\label{sec:name}
\addcontentsline{toc}{section}{de temps}

\textit{A COMPLETER}


%%%%%%%%%%%%%%%%%%%%%%%%%%%%%%%%%%%%%%%%%%%%%%%%%%%%%%%%%%%%%%%%%%
%                                                                %
%                          ORGANISATION                          %
%                                                                %
%%%%%%%%%%%%%%%%%%%%%%%%%%%%%%%%%%%%%%%%%%%%%%%%%%%%%%%%%%%%%%%%%%
\chapter{D\'eroulement du projet}

Ce projet se d\'ecoupe en 3 phases.\\

\begin{enumerate}[{Phase}-I ]
\item \textit{\textbf{Recherche Documentaire et Prise en main}}.
\item \textit{\textbf{Recherche d'un bio-algorithme et d\'eveloppement}}.
\item \textit{\textbf{Impl\'ementation d'un modele de robot avec les kilobots}}.
\end{enumerate}

\medskip
\section*{Organisation}\label{sec:name}
\addcontentsline{toc}{section}{Organisation}

\textit{A COMPLETER}

\section*{Mat\'eriel Utilis\'e}\label{sec:name}
\addcontentsline{toc}{section}{Mat\'eriel Utilis\'e}

Le mat\'eriel se compose des kilobots, de transmetteur et d'un PC quelqu'il soit.

\section*{Logiciel Utilis\'e}\label{sec:name}
\addcontentsline{toc}{section}{Logiciel Utilis\'e}

Nous utiliserons les logiciels fournis pour l'API Kilobotics pour le d\'eveloppement de notre API, notre gestionnaire de version sera Github.\\
Le projet sera h\'eberg\'e sous l'organisation \textit{\textbf{LSDev8}}\footnote{https://github.com/LSDev8/}.\\
Nous ferons nos simulations sous le logiciel \textit{\textbf{V-Rep}}\footnote{http://www.coppeliarobotics.com} afin de valider de mani\`ere plus rapide nos algorithmes.\\

\section*{Cas d'utilisation}\label{sec:name}
\addcontentsline{toc}{section}{Cas d'utilisation}

\textit{A COMPLETER}

\end{document}
