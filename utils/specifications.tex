\documentclass[a4paper,8pt]{report}

\usepackage[utf8]{inputenc}
\usepackage[latin1]{inputenc}
\usepackage[francais]{babel}
%\usepackage[T1]{fontenc}

\usepackage[left=2cm]{geometry}
\usepackage{amsmath,amssymb,mathrsfs}
\usepackage{graphicx}
\usepackage[table,xcdraw]{xcolor}

\usepackage{enumerate}

\usepackage{pgfplots}
\usetikzlibrary{shapes,positioning}
\usepackage{tikz}

\usepackage[nottoc, notlof, notlot]{tocbibind}
\usepackage{verbatim}

%\tikzset{state/.style={circle,draw=black, very thick,minimum size=4em}}

%Utilisation des codes sources en C
\usepackage{listings}
\lstset{
  language=C,
  language=Bash,
  basicstyle=\footnotesize,
  numbers=left,
  numberstyle=\normalsize,
  numbersep=7pt,
}

\begin{document}

\begin{titlepage}
\newcommand{\HRule}{\rule{\linewidth}{0.5mm}} % Defines a new command for the horizontal lines, change thickness here
\center % Center everything on the page
 
%----------------------------------------------------------------------------------------
%HEADING SECTIONS
%----------------------------------------------------------------------------------------

\textsc{\LARGE Universit\'e Pierre et Marie Curie}\\[1.5cm] % Name of your university/college
\textsc{\Large Projet SAR}\\[0.5cm] % Major heading such as course name
%\textsc{\large Minor Heading}\\[0.5cm] % Minor heading such as course title

%----------------------------------------------------------------------------------------
%TITLE SECTION
%----------------------------------------------------------------------------------------

\HRule \\[0.4cm]
{ \huge \bfseries Kilobot}\\[0.4cm] % Title of your document
\HRule \\[1.5cm]
 
%----------------------------------------------------------------------------------------
%AUTHOR SECTION
%----------------------------------------------------------------------------------------

\begin{minipage}{0.4\textwidth}
\begin{flushleft} \large
\emph{Auteurs:}\\
Arnaud \textsc{Guermont} % Your name
\\Benjamin \textsc{Bielle} % Your name
\end{flushleft}
\end{minipage}
~
\begin{minipage}{0.4\textwidth}
\begin{flushright} \large
\emph{Superviseur:} \\
Dr. Swan \textsc{Dubois} % Supervisor's Name
\end{flushright}
\end{minipage}\\[4cm]

% If you don't want a supervisor, uncomment the two lines below and remove the section above
%\Large \emph{Author:}\\
%John \textsc{Smith}\\[3cm] % Your name

%----------------------------------------------------------------------------------------
%DATE SECTION
%----------------------------------------------------------------------------------------

{\large \today}\\[3cm] % Date, change the \today to a set date if you want to be precise

%----------------------------------------------------------------------------------------
%LOGO SECTION
%----------------------------------------------------------------------------------------

\includegraphics[width=7cm]{images/upmc.png}\\[1cm] % Include a department/university logo - this will require the graphicx package
 
%----------------------------------------------------------------------------------------

\vfill % Fill the rest of the page with whitespace
\end{titlepage}

\renewcommand{\contentsname}{Sommaire}
\tableofcontents
\listoffigures

%%%%%%%%%%%%%%%%%%%%%%%%%%%%%%%%%%%%%%%%%%%%%%%%%%%%%%%%%%%%%%%%%%
%                                                                %
%                          INTRODUCTION                          %
%                                                                %
%%%%%%%%%%%%%%%%%%%%%%%%%%%%%%%%%%%%%%%%%%%%%%%%%%%%%%%%%%%%%%%%%%
\chapter{Introduction}

\noindent \textbf{Ma\^itrise d'oeuvre :} DUBOIS Swan.\\
\noindent \textbf{Ma\^itrise d'ouvrage :} GUERMONT Arnaud, BIELLE Benjamin.\\

\section*{Contexte}\label{sec:name}
\addcontentsline{toc}{section}{Contexte}

\textbf{Kilobot} est une plate-forme constitu\'ee de petits robots autonomes. Cette plate-forme est propos\'ee par l'universit\'e d'Harvard\footnote{http://www.eecs.harvard.edu/ssr/projects/progSA/kilobot.html}. \\
Ces robots se d\'eplacent uniquement par vibration et communiquent par infrarouges.\\

\begin{figure}[!h]
    \centering
    \subfigure{ \includegraphics[width=6cm]{images/kilobotI.jpg}}
    \subfigure{ \includegraphics[width=6cm]{images/kilobotII.jpg}}
    \caption{Robots Kilobot}
\end{figure}

\smallskip
Le but de ce projet est d'impl\'ementer une \textbf{API} permettant de manipuler des mod\`eles robotiques sur la plate-forme Kilobot. Ces mod\`eles ne sont pas pr\'evus pour la plate-forme Kilobot, notre API devra rendre cette impl\'ementation possible en restant simple et si possible facilement extensible.\\

\bigskip

\section*{L'existant}\label{sec:name}
\addcontentsline{toc}{section}{L'existant}

Il existe deux API de base pour la plate-forme Kilobot.\\

\begin{itemize}
\item \textit{\textbf{K-team}}\footnote{http://www.k-team.com/mobile-robotics-products/kilobot}.
\item \textit{\textbf{Kilobotics}}\footnote{https://www.kilobotics.com/}.
\end{itemize}

\medskip
Celle propos\'ee par \textit{\textbf{K-team}} (fournie par d\'efaut avec les robots) nous permet une moins grande libert\'e que celle de \textit{\textbf{Kilobotics}}.\\
Donc notre \textit{\textbf{API}} se basera sur celle-ci car elle nous garantit une plus grande libert\'e, portabilit\'e et extensibilit\'e.

\section*{Veille Technologique}\label{sec:name}
\addcontentsline{toc}{section}{Veille Technologique}

Il existe plusieurs projets similaires aux kilobots, par exemple le projet \textit{\textbf{I-Swarm}}\footnote{http://www.hizook.com/blog/2009/08/29/i-swarm-micro-robots-realized-impressive-full-system-integration}, mais seule la plate-forme \textit{\textbf{Kilobot}} permet de manipuler des centaines de robots de mani\`ere r\'ealiste hors d'un simulateur (avec un prix convenable).\\
Le projet \textit{\textbf{E-Swarm}}\footnote{http://www.e-swarm.org/research_main.php} propose un plate-forme int\'eressante mais le prix de chaque robot en fait une plate-forme non \'eligible pour ce projet.\\

\begin{figure}[!h]
    \centering
    \subfigure{ \includegraphics[width=6cm]{images/iSwarm.jpg}}
    \subfigure{ \includegraphics[width=6cm]{images/iSwarmII.jpg}}
    \caption{Robots I-Swarm}
\end{figure}

\begin{figure}[!h]
    \centering
    \subfigure{ \includegraphics[width=10cm]{images/eSwarmII.jpg}}
    \subfigure{ \includegraphics[width=10cm]{images/eSwarmIII.jpg}}
    \caption{Robots E-Swarm}
\end{figure}

%%%%%%%%%%%%%%%%%%%%%%%%%%%%%%%%%%%%%%%%%%%%%%%%%%%%%%%%%%%%%%%%%%
%                                                                %
%                          CONTRAINTES                           %
%                                                                %
%%%%%%%%%%%%%%%%%%%%%%%%%%%%%%%%%%%%%%%%%%%%%%%%%%%%%%%%%%%%%%%%%%
\chapter{Contraintes}

\section*{Mat\'erielles}\label{sec:name}
\addcontentsline{toc}{section}{Mat\'erielles}

Le but du mod\`ele est de repr\'esenter une grande quantit\'e de robot (presque un essaim) pour pouvoir effectuer des t\^aches collectives qui seraient impossible \`a r\'ealiser en solitaire. De nombreux exemples dans la nature mette en action ce principe : les fourmis peuvent explorer des zones d'un p\'erim\`etres de plusieurs kilom\`etres par exemple. \\
Dans la robotique, il existe des algorithmes divers telle-que l'auto-assemblage, les constructions collectives ou encore l'exploration.\\ 
Ces algorithmes sont fait pour \^etre test\'e sur plusieurs centaines voir milliers d'individus. Cependant, pour des raisons de co\^uts, de temps ou encore de complexit\'e, il est parfois uniquement possible de les valid\'es en simulation. \\
C'est pourquoi un mod\`ele comme celui que les kilobots impl\'emente fut cr\'ee. Les robots devaient avoir plusieurs capacit\'es : communiquer et percevoir leur environnement par l'interm\'ediaire de capteurs, conna\^itre la distance entre eux et se d\'eplacer en avant, tourner sur la gauche et sur la droite. De plus, ils devaient aussi pouvoir tous \^etre contr\^ol\'e ensemble simultan\'ement par un seul op\'erateur. Mais la principale contrainte restait le co\^ut. En effet, il \'etait tr\`es dur auparavant de tester des algorithmes sur des essaims de robots qui co\^utait plusieurs centaines de dollars \`a l'unit\'e.\\

\subsection*{Communication et perception de l'environnement}\label{subsec:name}
\addcontentsline{toc}{subsection}{Communication et perception de l'environnement}

De nombreux algorithmes collectifs utilise la communication entre robots et la distance entre ces m\^eme robots pour piloter individuellement le comportement de chaque robot. Il est donc indispensable que les kilobots puissent communiquer entre eux ainsi que mesurer la distance qui les s\'epares l'un de l'autre. \\
Pour qu'ils puissent communiquer, chaque kilobot est pourvus d'une led pour transmettre et d'une photodiode infrarouge comme r\'ecepteur qui sont tous les deux plac\'es en-dessous au centre du robot. En \'etant plac\'e au centre, le r\'ecepteur peut recevoir des messages d'une \'egale distance dans toutes les directions. Les messages sont envoy\'es par la led et utilisent la r\'eflexion du support sur lequel ils \'evoluent pour transmettre (c'est pourquoi le support doit \^etre r\'efl\'echissant comme un tableau blanc ou un miroir par exemple). En utilisant cette technique, un kilobot peut transmettre dans un rayon de 7 centim\`etres \`a un d\'ebit de 30 kb/s. La m\'ethode d'acc\`es CSMA/CA est utilis\'es pour \'eviter les collisions (tous les robots \'emettent sur le m\^eme canal).\\
Une led RGB est aussi situ\'e sur le c\^ot\'e pour pouvoir informer l'utilisateur.\\

\medskip
Les kilobots sont aussi pourvus d'un capteur de lumi\`ere permettant de mesurer l'intensit\'e lumineuse de l'environnement du robot.\\

\subsection*{Mesure de la distance}\label{subsec:name}
\addcontentsline{toc}{subsection}{Mesure de la distance}

Le calcul de la distance entre un robot et un autre se fait via l'envoi et la r\'eception du message. L'intensit\'e de la lumi\`ere \'emise par la led diminue proportionnellement avec la distance qu'elle parcours. Ils ne restent alors plus qu'au r\'ecepteur \`a calculer la distance en fonction de l'intensit\'e du signal re\c cu. Comme le signal peut \^etre affect\'e par l'environnement, la pr\'ecision est de +/- 2 mm.\\

\subsection*{D\'eplacement}\label{subsec:name}
\addcontentsline{toc}{subsection}{D\'eplacement}

L'une des principales capacit\'es des robots est de pouvoir se mouvoir dans leur environnement. L'une des solutions les plus simples et efficace est d'utiliser un design \`a deux roues, chaque roue \'etant contr\^ol\'e par un moteur. Mais cette solution est aussi tr\`es ch\`ere et ne permettra pas de r\'ealiser un robot \`a bas co\^uts. \\
Pour y rem\'edier, les kilobots disposent de trois jambes rigides et de deux moteurs \`a vibration dispos\'e de chaque c\^ot\'es du robot. Lorsqu'un des moteurs est activ\'es, il g\'en\`ere une force centrip\`ete qui est convertis en une force vers l'avant transmise \`a la jambe situ\'e en dessous du moteur. Gr\^ace au positionnement d\'ecentr\'e de chaque moteur, la mise en marche de l'un ou l'autre va causer une rotation vers la gauche ou la droite en fonction du moteur. En activant les deux moteurs en m\^eme temps et en contr\^olant la magnitude de leur vibration, ils est alors possible de faire avancer le robot en ligne droite vers l'avant. Il peut ainsi se d\'eplacer \`a une vitesse d'environs 1 cm/s et pivoter \`a environs 45°/s.\\

\subsection*{Contr\^ole des robots par l'op\'erateur}\label{subsec:name}
\addcontentsline{toc}{subsection}{Contr\^ole des robots par l'op\'erateur}

Lorsqu'on utilise un tr\`es grand nombre de robots, le chargement de leur programme, leur mise en marche ou encore le rechargement de leur alimentation peut vite prendre beaucoup de temps pour l'utilisateur : \'Emettons l'hypoth\`ese qu'il faille 3 secondes pour allumer un robot via un interrupteur. Pour un groupe de 1000 robots, cela prendrais au total 50 minutes, ce qui est bien trop long pour \^etre efficace. \\
Les kilobots sont donc con\c cus pour \^etre utilis\'e \`a grande \'echelle par une seule personne.\\

\medskip
La communication entre l'op\'erateur et les robots passent par un contr\^oleur. Le contr\^oleur est une plate-forme disposant d'une connexion usb pour interagir avec l'ordinateur de l'utilisateur et de multiples led pour envoyer des messages aux kilobots. Chaque kilobot dispose sur son dos d'un r\'ecepteur pour la r\'eception des signaux. Le contr\^oleur est l'interface entre l'op\'erateur et l'essaim de robot. Dispos\'e \`a une hauteur d'un m\`etre, il couvre un p\'erim\`etre d'un m\`etre. Il permet d'effectuer multiples op\'eration comme la mise en veille, le r\'eveil, le chargement d'un nouveau programme ou encore l'arr\^et du programme en cours d'ex\'ecution. Il est aussi utilis\'e pour la calibration des moteurs des kilobots.\\

\medskip
Le probl\`eme \'evoqu\'e auparavant aurait aussi pu s'appliquer aux kilobots. Pour y rem\'edier, les robots disposent d'un mode veille. Le microcontr\^oleur peut d\'esactiver le r\'egulateurs de tension pour les moteurs et aussi celui pour le syst\`eme de communication (led et r\'ecepteur). Il ne reste plus que celui pour le microcontr\^oleur principale. Dans cet \'etat de veille, le microcontr\^oleur va tester toutes les minutes pendant 10 ms si il a re\c cu un message de r\'eveil. Si c'est le cas, il r\'ealimente alors les r\'egulateurs de tension et le robot redevient de nouveau op\'erationnel. Ce mode veille permet \`a un kilobot d'avoir une autonomie de 3 mois. Mettre en veille un groupe ne prend que quelque seconde et le r\'eveil s'effectue en moins d'une minute. Les kilobots disposent aussi d'un interrupteur coupant toute alimentation du robot si ces derniers venait \`a ne pas \^etre utilis\'es pendant une longue p\'eriode.\\

\medskip
Le rechargement de l'alimentation des robots est aussi un facteur de temps. Les kilobots disposent d'une batterie au lithium-ion de 3.4 V 160 mAh. Elle leurs assurent une autonomie entre 3 et 24 heures selon l'usage qui en est fait. Le rechargement de celle-ci est aussi adapt\'e \`a la manipulation d'un grand nombre de robot. Pour cela, les kilobots utilisent deux de leurs trois jambes comme conducteur : L'utilisateur place les jambes des robots en contact avec un courant \'electrique de 6V via une plaque m\'etallique ou un tube m\'etallique et connecte le crochet situ\'e sur le dessus du robot \`a la masse. Cela permet le rechargement de plusieurs kilobots simultan\'ement sans intervention sur le robot. Par exemple, l'utilisation d'une plaque m\'etallique comme chargeur et d'une autre plaque qui viendrait se poser au dessus des kilobots est une solution abordable et adaptable \`a tr\`es grande \'echelle. Lorsque le chargement est termin\'e, la led RGB des kilobots en informe l'utilisateur.\\

\medskip
Le chargement d'un programme s'effectue via le contr\^oleur. Lorsque l'op\'erateur souhaite charger un nouveau programme, il envoi un message au kilobots pour qu'ils placent leur pointeur de programme dans le bootloader. Ce dernier dispose d'un programme qui va alors recevoir le nouveau programme de l'utilisateur et le placer dans la m\'emoire primaire. Le bootloader va ensuite v\'erifier que ce programme ne dispose pas d'erreur puis ensuite red\'emarrer le kilobots qui va alors ex\'ecuter le nouveau programme. Cette m\'ethode permet de programmer un essaim de robot en environs 35 secondes. \\

\subsection*{Co\^uts}\label{subsec:name}
\addcontentsline{toc}{subsection}{Co\^uts}

La principale contrainte lorsqu'on impl\'emente un mod\`ele \`a large \'echelle est le co\^uts. En effet, les robots actuelles d\'epasse les centaines de dollars \`a l'unit\'e et ne sont donc pas envisageable. \\
Les kilobots ont \'et\'e con\c cus dans le but de ne pas \^etre ch\`ere \`a construire. Le co\^ut \`a l'unit\'e pour 1000 robots est de 14 dollars, ce qui est environs dix fois moins que les autres robots con\c cus pour les algorithmes collectifs. \\
Pour cela, les robots disposent d'un moyen de locomotion par vibration comme vu auparavant et sont constitu\'e de pi\`ece simple et peut co\^uteuse : \\

\begin{itemize}
\item D\'eplacement : 3,12 \$
\item Alimentation : 3,61 \$
\item Communication/Capteur : 2,20 \$
\item Contr\^oleur : 2,83 \$
\item Structure : 1,55 \$
\item Divers : 0,74 \$
\item Total : 14,05 \$
\end{itemize}

\medskip
L'assemblage des robots est aussi tr\`es simple et rapide (la plus part des composants sont mont\'es \`a la  surface du PCB) et peuvent \^etre r\'ealis\'e par un robot pour un co\^ut d'environs 5 \$ par robot (pour 1000 robots). Cela prend donc environs 5 minutes pour obtenir un robot fonctionnel \`a partir des composants.\\

\section*{Logicielles}\label{sec:name}
\addcontentsline{toc}{section}{Logicielles}

\subsection*{Simulation}\label{subsec:name}
\addcontentsline{toc}{subsection}{Simulation}

Lorsqu'on impl\'emente des algorithmes collectifs, on d\'esire parfois les simuler avant de pouvoir les tester grandeur nature sur les robots (ici les kilobots) afin d'\'eviter les erreurs d'impl\'ementation ou de se retrouver avec un programme qui impl\'emente mal l'algorithme. \\
Pour cela, nous avons \`a notre disposition le simulateur V-Rep de la soci\'et\'e Coppelia Robotics qui est un simulateur sp\'ecialis\'e dans le domaine de la robotique. Il impl\'emente des mod\`eles 3d de l'ensemble des robots existants et il int\`egre aussi un cr\'eateur de mod\`ele 3d afin de pouvoir simuler des prototypes. Il dispose aussi d'un moteur physique permettant de v\'erifier la viabilit\'e de certains robots. Il utilise un langage de script (le Lua) pour programmer les robots et ainsi les faire \'evoluer dans l'environnement simul\'e. Il peut aussi \^etre coupl\'e avec de nombreuses interfaces via des API remotes.\\

\medskip
Le mod\`ele 3d des kilobots fut impl\'ement\'es par la K-Team dans le simulateur et permet la simulation de l'ensemble des fonctionnalit\'es des robots. Il existe deux impl\'ementations : avec et sans capteur de lumi\`ere. En effet, la simulation du capteur de lumi\`ere n\'ecessite des calculs suppl\'ementaires et allonge la dur\'ee de la simulation. Il n'est donc pas utile qu'il soit utiliser si l'algorithme n'en n'a pas besoin. \\
Le contr\^oleur est aussi simul\'e dans V-Rep via son propre mod\`ele 3d. Il permet de lancer le programme, arr\^eter le programme en cours et aussi de conna\^itre le niveau de batterie simul\'e. \\
L'ensemble de l'API de la K-Team est impl\'ement\'e dans le simulateur via un squelette de code inclus dans chaque script associ\'e \`a un objet kilobot.\\

\medskip
Malheureusement, le simulateur V-Rep ne r\'epond pas \`a nos besoins pour la simulation d'algorithme collectif sur les kilobots. Plusieurs probl\`eme furent rencontr\'es : \\ 

\medskip
Le langage utilis\'e pour les script est le Lua. Bien que tr\`es similaire au langage C, il nous oblige \`a retranscrire chaque programme en Lua. De plus, le fichier contenant le script en lua est incorpor\'e dans le fichier contenant la sc\`ene du simulateur (la sc\`ene comprend les mod\`ele de robot simul\'e, l'environnement ainsi que l'ensemble des scripts). C'est \`a la fois un avantage (partage d'une simulation tr\`es facile : tout est regroup\'e dans un seul fichier) et un inconv\'enient (oblig\'e de partager l'ensemble de la sc\`ene pour pouvoir travailler en groupe et ne pas pouvoir travailler sur le script sans installer ou ex\'ecuter le simulateur).\\

\medskip
Le simulateur dispose lors de la simulation de plusieurs vitesse de simulation. Il est alors utile de pouvoir acc\'el\'erer le temps pour voir si l'impl\'ementation de l'algorithme est robuste et stable. Sauf que l'augmentation de la vitesse de simulation engendre des bugs dans le d\'eroulement des scripts et ne permet pas de s'assurer du bon fonctionnement d'un programme de mani\`ere fiable.\\

\medskip
Enfin, V-Rep est un simulateur orient\'e m\'ecanique. Il dispose en effet d'un moteur physique et d'un environnement en 3 dimensions pour pouvoir simuler des robots complexes comme des bras robotiques. Nous l'utilisons pour effectuer des simulations d'algorithmes, donc comme un simulateur logique. Il n'est alors pas adapt\'e \`a nos besoins (trop lourd, langage de script trop contraignant, nombreux bug durant la simulation).\\

\subsection*{Firmware}\label{subsec:name}
\addcontentsline{toc}{subsection}{Firmware}

Les kilobots sont un mod\`ele de robot d\'evelopp\'es par l'universit\'e d'Harvard. L'entreprise suisse K-Team Corporation les proposes \`a la vente et s'occupe de la fabrication et de l'assemblage. Ils fournissent les robots avec leur propre firmware d\'evelopp\'e en interne par la K-Team. \\
Ce firmware poss\`ede un inconv\'enient : il n'est compatible uniquement sous windows. \\
En effet, pour que l'op\'erateur puisse charger le programme dans les kilobots, il doit passer par le contr\^oleur qui est l'interface entre l'ordinateur et les robots. Or, le driver du contr\^oleur est uniquement disponible pour Windows et la K-Team ne fournit pas les sources pour un portage sous Linux. \\
De plus, nous avons rencontr\'es des probl\`emes lors de la calibration des moteurs avec le firmware d'origine. Ce dernier ne permet la calibration uniquement du moteur gauche ou du moteur droit mais pas des deux lorsqu'on souhaite avancer. Or, le positionnement des jambes de chaque kilobot n'\'etant pas parfaitement identique, nous avons besoin de valeurs diff\'erentes pour les deux moteur afin que le robot se d\'eplace en ligne droite.\\

\medskip
Apr\`es recherche, nous d\'ecid\^ames d'utiliser le firmware kilobotics. Kilobotics est le nom de la librairie d\'evelopp\'e par l'universit\'e d'Harvard. Il est open-source et est plus r\'ecent et dispose d'un driver pour Linux. De plus, son utilisation limite les risques pour le mat\'erielle. Le contr\^oleur permet de charger les programmes dans les kilobots. Mais pour effectuer cela avec le firmware d'origine de la K-Team, il faut \`a chaque nouveau programme reprogrammer le microcontr\^oleur du contr\^oleur, ce qui implique le risque de se retrouver avec un microcontr\^oleur mal programmer et alors un contr\^oleur potentiellement hs. Le firmware kilobotics n\'ecessite lui de ne le reprogrammer qu'une seule fois, ce qui limite grandement les risques. \\
En ce qui concerne la calibration, kilobotic permet via l'interface graphique (kilogui) de calibrer chaque moteur s\'epar\'ement mais aussi de leur attribuer des valeurs diff\'erentes pour le d\'eplacement en ligne droite. Kilogui propose aussi la possibilit\'e de rentrer directement les valeurs souhait\'es (entre 1 et 255) pour la puissance des moteurs. Enfin, il est possible d'affecter un identifiant \`a chaque robot qui sera stock\'e dans la m\'emoire EPROM.\\

%\section*{de temps}\label{sec:name}
%\addcontentsline{toc}{section}{De temps}

%\textit{A COMPLETER}

%%%%%%%%%%%%%%%%%%%%%%%%%%%%%%%%%%%%%%%%%%%%%%%%%%%%%%%%%%%%%%%%%%
%                                                                %
%                           LE PRODUIT                           %
%                                                                %
%%%%%%%%%%%%%%%%%%%%%%%%%%%%%%%%%%%%%%%%%%%%%%%%%%%%%%%%%%%%%%%%%%
\chapter{L'API}

\section*{Description des attentes}\label{sec:name}
\addcontentsline{toc}{section}{Description des attentes}

Le projet doit permettre de cr\'eer une API simulant un autre type de mod\`ele robotique sur la plate-forme Kilobot.\\
Ce mod\`ele se nomme : \textbf{mod\`ele CORDA}.\\
Le mod\`ele CORDA nous impose quelques contraintes, ces contraintes portent sur l'aspect physique des robots, on peut les r\'esumer de la fa\c con suivante : \\

\begin{itemize}
\item chaque robot poss\`ede un rep\`ere orthonorm\'e.
\item chaque robot poss\`ede une vision "infinie".
\item chaque robot ne poss\`ede aucun moyen de communiquer avec ces cong\'en\`eres.
\item chaque deplacement du robot est pr\'ecis.
\end{itemize}

\medskip
Ces contraintes posent plusieurs probl\`ematiques. En effet la plate-forme Kilobot ne propose aucune des ces contraintes. (cf chapitre pr\'ec\'edent)\\
Donc comment simuler une vision "infinie" avec la plate-forme Kilobot ? Comment simuler un axe orthonorm\'e ? Comment localiser les robots les uns par rapport aux autres ? Et plus important comment un robot peut se localiser dans l'espace ?\\

%%%%%%%
% IMAGES du modele theorique.
%%%%%%%

\section*{Impl\'ementation}\label{sec:name}
\addcontentsline{toc}{section}{Impl\'ementation}

Nous pouvons r\'esoudre le probl\`eme de la localisation dans l'espace (ainsi que la localisation par rapport aux autres) et de l'axe orthonorm\'e en utilisant la m\'ethode de la trilat\'eration. Les robots Kilobot ne poss\`edent pas de cam\'era et ont une port\'ee de communication limit\'ee ce qui nous ne permet pas d'impl\'ementer un vision infinie \`a proprement parler.\\

\subsection*{Premi\`ere Approche}\label{sec:name}
\addcontentsline{toc}{subsection}{Premi\`ere Approche}

Nous pouvons, de mani\`ere na\"{i}ve, r\'esoudre le prob\`eme de localisation des robots en disposant des balises sur un cercle de diam\`etre 14cm (cf figure : ~\ref{figureI}).\\
Chaque point de la figure repr\'esente une balise, et les cercles permettents de visualiser les zones de communications.\\
Nous remarquons que l'ajout de cinq balises sur le cercle nous permet pas de recouvrir de mani\`ere optimale la zone (int\'erieur du cercle) donc nous doublons le nombre de balises.\\
Cela permet effectivement de couvrir toute la zone et de localiser de mani\`ere "pr\'ecise".\\
Notre approche r\'esout le probl\`eme de la localisation mais pose un autre probl\`eme. Comment augmenter la zone de travail?\\
En effet, la zone couverte permet de faire fonctionner correctement seulement deux kilobots.\\
Donc notre approche n'est pas viable pour simuler le mod\`ele CORDA sur la plate-forme Kilobot.\\

\begin{figure}[!h]
    \centering
    \subfigure{ \includegraphics[width=14cm]{images/locate_bounding_box_demo.jpg}}
    \caption{Simulation de la première approche}
\end{figure}

%%%%%%%
% IMAGES des robots avec la premiere approche
%%%%%%%

\begin{figure}
  \begin{tikzpicture}
    %%%%%%%%%%%%%%%%%%%%%%%%%%%%%%%%%%%%%%%
    %           VARIABLE RAYON            %
    %%%%%%%%%%%%%%%%%%%%%%%%%%%%%%%%%%%%%%%
    \newcommand{\rayon}{4}

    %%%%%%%%%%%%%%%%%%%%%%%%%%%%%%%%%%%%%%%
    %        AVEC 5 BALISES               %
    %%%%%%%%%%%%%%%%%%%%%%%%%%%%%%%%%%%%%%%
    \draw[line width=3pt] (0,0) -- (0,\rayon);                % 0 -> A
    \draw[red,line width=2pt] (0,\rayon) -- (18:\rayon);      % A -> B
    \draw[red,line width=2pt] (18:\rayon) -- (-54:\rayon);    % B -> C
    \draw[red,line width=2pt] (-54:\rayon) -- (-126:\rayon);  % C -> D
    \draw[red,line width=2pt] (-126:\rayon) -- (-198:\rayon); % D -> E
    \draw[red,line width=2pt] (-198:\rayon) -- (0,\rayon);    % E -> A
    
    %%%%%%%%%%%%%%%%%%%%%%%%%%%%%%%%%%%%%%%
    %        AVEC 10 BALISES              %
    %%%%%%%%%%%%%%%%%%%%%%%%%%%%%%%%%%%%%%%
    \draw[blue,line width=2pt] (0,\rayon) -- (54:\rayon);       % A  -> AB
    \draw[blue,line width=2pt] (54:\rayon) -- (18:\rayon);      % AB -> B
    \draw[blue,line width=2pt] (18:\rayon) -- (-18:\rayon);     % B  -> BC
    \draw[blue,line width=2pt] (-18:\rayon) -- (-54:\rayon);    % BC -> C
    \draw[blue,line width=2pt] (-54:\rayon) -- (-90:\rayon);    % C  -> CD
    \draw[blue,line width=2pt] (-90:\rayon) -- (-126:\rayon);   % CD -> D
    \draw[blue,line width=2pt] (-126:\rayon) -- (-162:\rayon);  % D  -> DE
    \draw[blue,line width=2pt] (-162:\rayon) -- (-198:\rayon);  % DE -> E
    \draw[blue,line width=2pt] (-198:\rayon) -- (-234:\rayon);  % E  -> EA
    \draw[blue,line width=2pt] (-234:\rayon) -- (0,\rayon);     % EA -> A

    %%%%%%%%%%%%%%%%%%%%%%%%%%%%%%%%%%%%%%%
    %         ARC avec 5 balises          %
    %%%%%%%%%%%%%%%%%%%%%%%%%%%%%%%%%%%%%%%
    \draw[densely dashdotdotted,red] (0,\rayon) circle (\rayon);
    \draw[densely dashdotdotted,red] (18:\rayon) circle (\rayon);
    \draw[densely dashdotdotted,red] (-54:\rayon) circle (\rayon);
    \draw[densely dashdotdotted,red] (-126:\rayon) circle (\rayon);
    \draw[densely dashdotdotted,red] (-198:\rayon) circle (\rayon);

    %%%%%%%%%%%%%%%%%%%%%%%%%%%%%%%%%%%%%%%
    %         ARC avec 10 balises         %
    %%%%%%%%%%%%%%%%%%%%%%%%%%%%%%%%%%%%%%%
    \draw[densely dashdotdotted,blue] (54:\rayon) circle (\rayon);
    \draw[densely dashdotdotted,blue] (-18:\rayon) circle (\rayon);
    \draw[densely dashdotdotted,blue] (-90:\rayon) circle (\rayon);
    \draw[densely dashdotdotted,blue] (-162:\rayon) circle (\rayon);
    \draw[densely dashdotdotted,blue] (-234:\rayon) circle (\rayon);

    %%%%%%%%%%%%%%%%%%%%%%%%%%%%%%%%%%%%%%%
    %               INIT                  %
    %%%%%%%%%%%%%%%%%%%%%%%%%%%%%%%%%%%%%%%
    \draw[line width=3pt] (0,0) circle (\rayon);
    
    \draw (0,\rayon) node[above]{$A$};
    \draw (18:\rayon.2) node[above]{$B$};
    \draw (-54:\rayon.2) node[right]{$C$};
    \draw (-126:\rayon.2) node[below]{$D$};
    \draw (-198:\rayon.2) node[left]{$E$};
    
    \draw (54:\rayon.2) node[above]{$AB$};
    \draw (-18:\rayon.2) node[right]{$BC$};
    \draw (-90:\rayon.2) node[below]{$CD$};
    \draw (-162:\rayon.2) node[left]{$DE$};
    \draw (-234:\rayon.2) node[left]{$EA$};
    
  \end{tikzpicture}
  \caption{Premi\`ere Approche}
  \label{figureI}
\end{figure}

\subsection*{Seconde Approche}\label{sec:name}
\addcontentsline{toc}{subsection}{Seconde Approche}

La premi\`ere approche r\'esolvait le probl\`eme de localisation et de positionnement, mais ne permettait pas de faire fonctionner plus de deux robots de mani\`ere viable.\\
Donc notre deuxi\`eme approche devra, elle aussi, de resoudre le probl\`eme de localisation et de positionnement mais devra, par contre, agrandir la zone de travail des robots (par exemple de permettre \`a 6 robots ou plus d'\^etre dans la zone de travail sans se g\^ener).\\

\medskip
Nous pouvons utiliser la m\'ethode dite de "bounding box" mais cette m\'ethode augmenterait le nombre de balises par rapport au nombre de robots utiles de mani\`ere trop importante (du faite de la faible distance de communication des Kilobots).\\
Nous pouvons utiliser la m\'ethode de multilat\'eration qui ne demande que trois balises et ne demande pas une connectivit\'e directe avec les balises (la distance est estim\'ee gr\^ace \`a la propagation d'un gradient) mais cette m\'ethode pose le probl\`eme de la puissance de calcul, en effet la plate-forme Kilobot offre une puissance de calcul limit\'ee ne permettant pas d'impl\'ementer cette m\'ethode.\\

\smallskip
La solution serait de proposer une m\'ethode hybride entre ces deux m\'ethodes, soit la m\'ethode de "bounding box" avec la distance estim\'ee avec un gradient.\\
Cette solution permettra \`a terme d'agrandir la zone de travail mais aura comme cons\'equence une perte de pr\'ecision.\\


\section*{Fonctions de l'API}\label{sec:name}
\addcontentsline{toc}{section}{Fonctions de l'API}

L'API devra fonctionner sur tous types de de syst\`emes (\textit{Windows}, \textit{OSX}, \textit{Linux}) et doit \^etre extensible afin de facilit\'e l'ajout de futures fonctionnalit\'ees.\\

La premi\`iere version de l'API ne proposera que les m\'ethodes suivantes :

\begin{description}
\item[getPosition] qui permet \`a un robot de connaitre sa position.
\item[getVision] qui permet \`a un robot de connaitre la position des robots qui l'entoure.
\item[toPosition] qui permet \`a un robot de se rendre aux coordonn\'ees en param\`etre.
\end{description}

%%%%%%%%%%%%%%%%%%%%%%%%%%%%%%%%%%%%%%%%%%%%%%%%%%%%%%%%%%%%%%%%%%
%                                                                %
%                          ORGANISATION                          %
%                                                                %
%%%%%%%%%%%%%%%%%%%%%%%%%%%%%%%%%%%%%%%%%%%%%%%%%%%%%%%%%%%%%%%%%%
\chapter{D\'eroulement du projet}

Ce projet se d\'ecoupe en 3 phases.\\

\begin{enumerate}[{Phase}-I ]
\item \textit{\textbf{Recherche Documentaire et Prise en main}}.
\item \textit{\textbf{Recherche d'un bio-algorithme et d\'eveloppement}}.
\item \textit{\textbf{Impl\'ementation d'un modele de robot avec les kilobots}}.
\end{enumerate}

\medskip

\section*{Phase I}\label{sec:name}
\addcontentsline{toc}{section}{Phase I}

La phase I ne comprenant que de la recherche documentaire et la prise en main de la plate-forme, la r\'epartition se deroula ainsi, Arnaud prit en main la plate-forme et Benjamin la recherche documentaire.\\
Pour la prise en main de la plate-forme, deux algorithmes fut choisis mettant chacun en scène une ou plusieurs capacités: la locomotion et la communication.\\
Le premier algorithme est celui de l'orbite. Il permet de déplacer de façon circulaire un kilobot autour d'un point représenter par un autre robot.\\
Le second met en avant la communication entre eux et se nomme firefly. Il réalise une synchronisation de plusieurs horloges (ici celle de plusieurs robots) via échange de messages.\\

\subsection*{Orbite}\label{subsec:name}
\addcontentsline{toc}{subsection}{Orbite}

L'orbite reprend le principe des satellites naturelles comme la lune qui tourne autour de la terre. Chaque astre prend ici la forme d'un kilobot. Le premier va effectuer une révolution autour du second pendant que ce dernier lui enverra des messages en continu. L'emploi d'un second kilobot est indispensable pour permettre à un autre d'avancer en formant un cercle. Sans point de repère, il est impossible d'en effectuer un correctement, les déplacements par vibrations étant trop imprécis et entraînant des mouvements non désirés. La surface sur lequel évolue le robot rajoute encore un facteur imprévisible.

\subsubsection{Principe de l'algorithme}\label{subsubsec:name}

Un kilobot va rester en position stationnaire et envoyer des messages en continue. Le contenu de ceux-ci est d'aucune importance. Ils ne servent qu'a transmettre la distance qui sépare l’émetteur des récepteurs.\\
Le second kilobot va lui avancer en continue et recevoir les messages du premier. A chaque réception, il va calculer la distance qui le sépare de l'émetteur et ainsi pouvoir faire des corrections de trajectoire. Il avance ainsi tout droit tant qu'il se trouve dans les bornes définis dans le programme. Si il se trouve à une distance inférieur à celle-ci, il tournera à gauche jusqu’à de nouveau être dans les bornes. Il effectuera la même opérations si il est à une distance supérieur, mais vers la droite.\\
Ainsi, il va naturellement réaliser une révolution autour du kilobot fixe en tournant autour.\\

\begin{center}
  \rule{\linewidth}{.5pt}
  \textbf{Orbite}\\
  \rule{\linewidth}{.5pt}
\end{center}
\underline{Legende} : \textbf{S} (Send) et \textbf{R} (Receive).\\

\begin{table}[h]
\begin{tabular}{lllll}
\textit{Master} &                          &  & \textit{Slave} &                                                                                                                                                                                                                                                          \\
                &                          &  &                &                                                                                                                                                                                                                                                          \\
\textbf{S :}    & envoie un message $<$$>$ &  & \textbf{S :}   &                                                                                                                                                                                                                                                          \\
\textbf{R :}    &                          &  & \textbf{R :}   & r\'eception d'un message $<>$                                                                                                                                                                                                               \\
                &                          &  &                & \begin{tabular}[c]{@{}l@{}} Si distance \textgreater= distance\_max\\ tourner à droite\\ avancer tout droit\\ Sinon SI distance \textless= distance\_min\\      touner à gauche \\ avancer tout droit\\ Sinon\\ avancer tout droit\\ Fin Si \end{tabular}
\end{tabular}
\end{table}

Le kilobot \textit{Master} reste immobile tandis que le kilobot \textit{Slave} orbite autour de lui.\\

\subsection*{Firefly}\label{subsec:name}
\addcontentsline{toc}{subsection}{Firefly}

L'algorithme firefly est un bio-algorithme s'inspirant du comportement des lucioles dans la nature. Quand elles se retrouvent en groupe, elles sont capables de "clignoter" de manière synchronisé.
Les kilobots font de même avec cet algorithme.\\

Pour permettre possible la synchronisation de plusieurs individus, nous implémentons une horloge logique dans chacun. La synchronisation va se réaliser en collectant les horloges contenus dans les messages des voisins afin de déterminer la moyenne du nombre de tick d'horloge qui sont en avance (ou en retard).\\

Le kilobot dispose donc d'un tableau dont les indexes sont le nombre de tick en retard ou en avance par rapport à ces voisins.\\
Exemple : Si la différence entre son horloge et l'horloge reçu dans un message est de 8 ticks, alors il incrémentera de 1 la valeur contenu à l'index 8 du tableau.\\
Le tableau est ainsi mis à jours à chaque réception d'un message seulement si la différence entre l'horloge de l’émetteur et celle du récepteur est inférieur à la moitié de la période d'horloge. Cette prévention permet d'éviter que deux kilobots voisins se mettent à jour en même temps.\\

A chaque itération de la boucle principale du programme, la moyenne de la somme des ticks enregistrés dans le tableau est effectué (tick\_offset) et celui-ci remis à zéro. La différence entre l'horloge local (kilo\_ticks) et celles reçus est calculé :\\

modulo\_clock = ((kilo\_ticks – tick\_offset) / 4) \% 32\\

Si modulo\_clock est égale à zéro, le robot peut alors faire clignoter sa led. 
Ainsi la synchronisation se fait au fil des cycles en réduisant l'écart entre chaque horloge.

\begin{center}
  \rule{\linewidth}{.5pt}
  \textbf{Luciole/Firefly}\\
  \rule{\linewidth}{.5pt}
\end{center}
\underline{Legende} : \textbf{S} (Send) et \textbf{R} (Receive).\\

\begin{table}[h]
\begin{tabular}{lllll}
\textbf{S :} & envoie un message $<$Horloge$>$ &  & \textbf{R :} & r\'eception d'un message $<$Horloge$>$                                                                                                                                                                             \\
           &                                   &  &            & \begin{tabular}[c]{@{}l@{}}Si mon\_Horloge \textgreater Horloge            \\     décrémentation de mon\_Horloge d'une demi-période\\ Sinon    \\     incrémentation de mon\_Horloge d'une demi-période\\ Fin Si\end{tabular} \\
\end{tabular}
\end{table}

\section*{Phase II}\label{sec:name}
\addcontentsline{toc}{section}{Phase II}

La phase II demandant plus de r\'eflexion, nous m\^imes nos comp\'etences acquises \`a la premi\`ere phase en commun afin de proposer un bio-algorithme digne d'int\'er\^et.\\
Notre bio-algorithme se base sur le principe de phototaxis.\\

\smallskip
\textit{R\'eaction de locomotion d'organismes mobiles provoqu\'ee par la lumi\`ere et qui les porte soit \`a s'en approcher (phototaxie positive), soit \`a s'en \'eloigner (phototaxie n\'egative).}\\

\smallskip
La premi\`ere version de notre algorithme ne permet pas une entraide des kilobots, c'est \`a dire que chaque robot suit son programme sans se souci\'e des autres.\\
Cette version ne permet donc pas a un essaim en formation de suivre la lumi\`ere tout en gardant cette formation.\\
L'algorithme se d\'eroule ainsi, chaque robot capte la lumi\`ere ambiante (la lumi\`ere ambiante doit \^etre faible afin de ne pas g\^ener les capteurs des kilobots dans la d\'etection de la source cible de lumi\`ere), si la source de lumi\`ere d\'etect\'ee par le capteur provient de la droite on tourne \`a droite et inversement, apr\`es chaque demi-tour on avance tout droit pendant un certain laps de temps.\\
Les kilobots ne communiquent jamais ensemble.\\

\newpage
\begin{center}
  \rule{\linewidth}{.5pt}
  \textbf{Phototaxis}\\
  \rule{\linewidth}{.5pt}
\end{center}

\begin{verbatim}
var currentLight <- indice de lumière ambiante.
var currentDir   <- indice de direction.

Tant que vrai   
     réception de la lumière

     Si currentLight > barrière haute
     alors
        -tourner à droite
        -currentDir <- droite;
     Fin Si
 
     Si currentLight < barrière basse
     alors
        -tourner à gauche
        -currentDir <- gauche;
     Fin Si
Fin Tant que
\end{verbatim}

%La seconce version de notre algorithme utilise l'algorithme de gradient afin de garder les kilobots en formation jusqu'\`a la source de lumi\`ere.\\
%Par exemple les kilobots ont une formation en A, et ils doivent garder cette forme jusqu'\`a la destination.\\


\section*{Phase III}\label{sec:name}
\addcontentsline{toc}{section}{Phase III}

%%%%%%%%%%%%%%%%%%%%%%%%%%%%%%%%%%%%%%%%%%%%%%%%%%%%%%%%%%%%%%%%%%
%                                                                %
%                          ANNEXE                                %
%                                                                %
%%%%%%%%%%%%%%%%%%%%%%%%%%%%%%%%%%%%%%%%%%%%%%%%%%%%%%%%%%%%%%%%%%
\chapter{Annexe}

\section*{Mat\'eriel Utilis\'e}\label{sec:name}
\addcontentsline{toc}{section}{Mat\'eriel Utilis\'e}

Le mat\'eriel se compose des kilobots, de transmetteur et d'un PC quelqu'il soit.

\section*{Logiciel Utilis\'e}\label{sec:name}
\addcontentsline{toc}{section}{Logiciel Utilis\'e}

Nous utiliserons les logiciels fournis pour l'API Kilobotics pour le d\'eveloppement de notre API, notre gestionnaire de version sera Github.\\
Le projet sera h\'eberg\'e sous l'organisation \textit{\textbf{LSDev8}}\footnote{https://github.com/LSDev8/}.\\
Nous utilisions (au d\'ebut du projet) pour nos simulations le logiciel \textit{\textbf{V-Rep}}\footnote{http://www.coppeliarobotics.com} mais celui-ci se r\'ev\'ela trop contraignant dans son utilisation donc nous opt\^ames pour le logiciel \textit{\textbf{kbsim}}\footnote{https://github.com/ajhalme/kbsim} car celui-ci est un simulateur exclusivement pour kilobot (une copie de ce logiciel se trouve aussi sur l'organisation \textit{\textbf{LSDev8}}).\\


\section*{Tutoriels}\label{sec:name}
\addcontentsline{toc}{section}{Tutoriels}

\subsection*{Installation du contr\^oleur}\label{sec:name}
\addcontentsline{toc}{subsection}{Installation du contr\^oleur}

Le contr\^oleur exploite un microcontr\^oleur Atmega m328p. Lors de la première connexion du contr\^oleur \`a un ordinateur Windows, Windows va automatiquement aller t\'el\'echarg\'e les drivers correspondant via windows update et les installer. Il n'y a pas d'autre op\'erations \`a faire. Sous Linux, aucun driver n'est n\'ecessaire.\\
Si vous utilisez le firmware original de la K-Team, il vous faudra alors installer pour Windows un driver suppl\'ementaire afin de faire communiquer l'interface graphique avec le contr\^oleur. Pour cela, suivez les instructions fournis dans le manuel de la K-Team.

\begin{center}
  \fcolorbox{red}{white}{
    \begin{minipage}{1.0\textwidth}
      Si le contr\^oleur n'est pas reconnus par windows, assurer vous qu'il appara\^it bien dans le gestionnaire de p\'eriph\'erique.\\ Il doit normalement appara\^itre comme un Serial Port (FTDI) et un AVRisp mkII (Jungo).
    \end{minipage}
  }
\end{center}

\subsection*{Flashage du contr\^oleur et des kilobots}\label{sec:name}
\addcontentsline{toc}{subsection}{Flashage du contr\^oleur et des kilobots}

Si vous d\'esirez passer au firmware kilobotics, il est n\'ecessaire d'utiliser le logiciel AvrDude. Ce logiciel d\'evelopper par Atmel permet de reprogrammer le microcontr\^oleur du contr\^oleur. Pour cela, connecter le contr\^oleur \`a un ordinateur via un cable usb, ouvrer un terminal et placez-vous dans le r\'epertoire contenant le nouveau firmware (au format .hex) puis assurez-vous que le cavalier du contr\^oleur est bien positionn\'e en mode interne (c'est la position par d\'efaut : c'est \`a dire sur la gauche). \\

\begin{figure}[!h]
    \centering
    \subfigure{ \includegraphics[width=8cm]{images/flashI.png}}
    \caption{Cavalier en position interne}
\end{figure}

Enfin, ex\'ecuter la commande suivante :\\

\begin{center}
  \textbf{avrdude -p m328  -P usb -c avrispmkII -U "flash:w:controller.hex:i"}
\end{center}

Pour flasher les kilobots, l'op\'eration est tr\`es similaire. Connecter le contr\^oleur, ouvrer un terminal et placez-vous dans le r\'epertoire contenant le nouveau firmware (toujours au format .hex). Placez cette fois-ci le cavalier du contr\^oleur en mode externe (sur la droite): \\

\begin{figure}[!h]
    \centering
    \subfigure{ \includegraphics[width=8cm]{images/flashII.png}}
    \caption{Cavalier en position externe}
\end{figure}

Puis connecter le c\^able au contr\^oleur et au kilobot via les 6 trous pr\'esent sur le c\^ot\'e du kilobot.\\
Enfin, ex\'ecuter la commande suivante :\\

\begin{center}
  \textbf{avrdude -p m328p -P usb -c avrispmkII -U "flash:w:bootloader.hex:i"}
\end{center}

\smallskip
\begin{center}
  \fcolorbox{red}{white}{
    \begin{minipage}{1.0\textwidth}
      Si AvrDude ne reconnaît pas le contr\^oleur, assurez vous d'avoir positionn\'e le cavalier dans la bonne position. Sinon, la connexion entre le connecteur du câble et le kilobot est tr\`es sensible. Il faut que toutes les broches du connecteur soit en contact avec le kilobot. N'h\'esitez surtout pas \`a presser l\'eg\`erement le connecteur contre le kilobot et \`a le placer l\'eg\`erement en biais afin d'assurer la bonne connexion entre le connecteur et le robot.
    \end{minipage}
  }
\end{center}

\newpage
\subsection*{Chargement des kilobots}\label{sec:name}
\addcontentsline{toc}{subsection}{Chargement des kilobots}

Les kilobots peuvent être rechargés via le chargeur de la K-Team. Pour cela, passez les en mode charge grâce au contrôlleur et à l'interface graphique kilogui, puis disposez les sur le chargeur. Ils clignoteront en rouge pendant la charge puis passeront au bleu une fois la charge terminée.

\begin{figure}[!h]
    \centering
    \subfigure{ \includegraphics[width=16cm]{images/kilobots_charging.jpg}}
    \caption{Kilobots sur le chargeur}
\end{figure}

\begin{center}
  \fcolorbox{red}{white}{
    \begin{minipage}{1.0\textwidth}
      Attention, en fonction de la version de kilogui dont vous disposez, il se peut que vous ne disposiez pas du boutton charge. Il existe deux version de kilogui: La version sous format .deb contient la calibration mais pas le mode charge. Il faut récuperer les sources de kilogui et les compilers vous-même.\\
      Attention, les sources de kilogui ne contiennent pas la calibration. Il faut donc utiliser les deux versions en fonction de l'action souhaité (calibration ou chargement).
    \end{minipage}
  }
\end{center}

\newpage
\section*{Codes Sources}\label{sec:name}
\addcontentsline{toc}{section}{Codes Sources}

Code source de la synchronisation "luciole/firefly".\\
\lstinputlisting[language=C]{../src/firefly.c}

Code source de l'orbit master.\\
\lstinputlisting[language=C]{../src/orbit_master.c}

Code source de l'orbit slave.\\
\lstinputlisting[language=C]{../src/orbit_slave.c}

%%%%%%%%%%%%%%%%%%%%%%%%%%%%%%%%%%%%%%%%%%%%%%%%%%%%%%%%%%%%%%%%%%
%                                                                %
%                          BIBLIOGRAPHIE                         %
%                                                                %
%%%%%%%%%%%%%%%%%%%%%%%%%%%%%%%%%%%%%%%%%%%%%%%%%%%%%%%%%%%%%%%%%%
\nocite{*}
\bibliographystyle{plain}
\bibliography{kilobot}

\end{document}
